\chapter{Compilación del código y ejecución}

Aquí están algunas guías para compilar y ejecutar el código fuente para este proyecto.

\section{Paquetes necesitados}

\subsection{Python}

Para instalar Python ver la página web \url{http://www.python.org}. Se recomienda la última versión. MacOS 10.5 viene con un paquete de Python adecuado y en Linux generalmente se puede usar el manejador de paquetes de la distribución para instalarlo si no está instalado por defecto.

\subsection{FUSE}

No se necesita instalar FUSE en Linux ya que es parte del kernel desde la versión 2.6.14. En MacOS X se recomienda usar MacFUSE (\url{http://code.google.com/p/macfuse/}). Ver el sitio web para más información. Al momento no existe una implementación completa para plataformas Windows.

\subsection{Módulos de Python}

Los siguientes módulos de Python son necesarios para ejecutar la aplicación:

\begin{enumerate}
\item setuptools
\item fuse-python
\item psyco
\item PyDbLite (\url{http://quentel.pierre.free.fr/PyDbLite/index.html})
\end{enumerate}

Para instalarlos se puede usar easy\_install (se instala automáticamente con Python) de la siguiente manera:

\begin{verbatim}
easy_install [nombre_del_paquete]
\end{verbatim}

Para PyDbLite existe una carpeta llamada \path{support/PyDbLite} donde se puede usar los mismos comandos que se presentan a continuación para instalarlo.

\section{Compilación e instalación}

Para compilar simplemente hay que ejecutar:

\begin{verbatim}
python setup.py build
\end{verbatim}

Para instalar:

\begin{verbatim}
sudo python setup.py install
\end{verbatim}

\section{Pruebas}

Existe un script en el directorio \path{test_scripts} que ejecuta estos comandos y además ejecuta los pruebas de unidad que existen. Para usar este script simplemente llamar:

\begin{verbatim}
test_scripts/build
\end{verbatim}

\subsection{Proctor}

Se recomienda Proctor (\url{http://www.doughellmann.com/projects/Proctor/}) para ejecutar las pruebas de la aplicación.


% ------------------------------------------------------------------------

%%% Local Variables: 
%%% mode: latex
%%% TeX-master: "../thesis"
%%% End: 
