\chapter{Compilación del código y ejecución}

Aquí están algunas guías para compilar y ejecutar el código fuente para este proyecto.

\section{Paquetes necesitados}

\subsection{Python}

Para instalar python ver la pagina web \url{http://www.python.org}. Se recomienda la última versión.

\subsection{FUSE}

No se necesita instalar fuse en linux ya que es parte del kernel desde la versión 2.6.14. En MacOS X se recomienda usar MacFUSE (\url{http://code.google.com/p/macfuse/}). Ver el sitio web para más información.

\subsection{Módulos de Python}

Los siguientes módulos de python son necesarios para ejecutar la aplicación:

\begin{enumerate}
\item python-setuptools
\item fuse-python
\item psyco
\item PyDbLite (\url{http://quentel.pierre.free.fr/PyDbLite/index.html})
\end{enumerate}

Para instalarlos se puede usar easy\_install (parte de python) para instalar estos paquetes con excepción de PyDbLite.

\subsection{Proctor}

Se recomienda Proctor (\url{http://www.doughellmann.com/projects/Proctor/}) para ejecutar los tests de la aplicación.

\section{Compilación e instalación}

Para compilar simplemente hay que ejecutar:

\begin{verbatim}
python setup.py build
\end{verbatim}

Para instalar:

\begin{verbatim}
sudo python setup.py install
\end{verbatim}

Existe un script en el directorio \path{test_scripts} que ejecuta estos comandos y además ejecuta los unit tests que existen. Para usar este script simplemente llamar:

\begin{verbatim}
test_scripts/build
\end{verbatim}


\section{Ejecución}

Una vez instalado el paquete se puede usar de la siguiente manera:

\begin{verbatim}
dejumble [punto_de_montaje] [opciones]
\end{verbatim}

Para más información sobre las opciones ejecutar:

\begin{verbatim}
dejumble --help
\end{verbatim}


% ------------------------------------------------------------------------

%%% Local Variables: 
%%% mode: latex
%%% TeX-master: "../thesis"
%%% End: 
