\def\baselinestretch{1}
\chapter{Conclusiones}
\ifpdf
    \graphicspath{{Conclusions/ConclusionsFigs/PNG/}{Conclusions/ConclusionsFigs/PDF/}{Conclusions/ConclusionsFigs/}}
\else
    \graphicspath{{Conclusions/ConclusionsFigs/EPS/}{Conclusions/ConclusionsFigs/}}
\fi

\section{Conclusiones generales}

Luego de implementar y poner a prueba este sistema se puede concluir que es factible la creación de sistemas de archivos proxy. El proceso de creación de nuevos módulos del sistema es sencillo ya que pueden ser escritos en muy pocas líneas. Por ejemplo el promedio de lineas de los filtros implementados es de 6.2 líneas de código y de 15 líneas para los organizadores. Los resultados de las pruebas son mixtos con resultados excelentes en las pruebas de creación de archivos con contenido y resultados buenos en la creación y borrado de múltiples archivos. Además se demostró que se pudo crear implementaciones completas de la funcionalidad de un sistema de archivos proxy.

\section{Resultados de creación y borrado de archivos}

En el caso de manejo de listas de archivos crecientes, el tiempo de respuesta se altera linealmente con respecto al tiempo de respuesta del sistema de archivos original, es decir crece en orden O(n). El tiempo que le toma al sistema de archivos proxy crear y borrar archivos es en promedio el doble que si se usa el sistema de archivos normal.

\section{Resultados de creación de archivos con contenido}

En el caso de crear archivos con contenido la penalización que se sufre por usar el sistema de archivos proxy en todos los casos fue prácticamente nula. Esto se debe a que todas las operaciones se las pasa directamente al sistema de archivos original sin alteraciones.

\section{Recomendaciones}

En un uso cotidiano de diferentes configuraciones del sistema de archivos proxy no es muy notable el tiempo que toma crear o borrar un archivo. Estas operaciones son en general muy pocas para usuarios normales. En cambio la mayoría de usuarios acceden a sus archivos muchas veces y cambian su contenido constantemente. 

Existen ciertas áreas del sistema de archivos que se modifican constantemente, creando y borrando archivos como por ejemplo los archivos temporales de internet. En este caso no sería práctico usar un sistema de archivos proxy en esa localización por dos razones. Primero, es un área que los usuarios normalmente no acceden directamente si no accede solamente el navegador y, segundo, el navegador espera una estructura especial para esa área y al usar un sistema de archivos proxy seguramente estaríamos cambiando esa estructura de directorios y archivos y el navegador no encontraría los archivos temporales. 

Sin embargo existen otras áreas del sistema de archivos que los usuarios acceden constantemente como son las carpetas donde se guardan documentos o imágenes. Estas carpetas pueden hacer uso de un sistema de archivos proxy para organizarse mejor. Habría que tomar una precaución que sería en el caso de que se desee copiar muchos archivos de un disco externo u otra localización a estas carpetas. En ese caso sería mejor desmontar temporalmente el sistema de archivos proxy para hacer la copia y luego volverlo a montar ya que en este caso si estaríamos creando muchos archivos y el sistema de archivos proxy puede afectar en el desempeño. 

En todo caso sería interesante poder hacer un estudio de usabilidad de los sistemas de archivo proxy con un grupo de usuarios no técnicos en el que se mida también patrones de uso del sistema de archivos en una sesión de usuario normal.



%%% ----------------------------------------------------------------------

% ------------------------------------------------------------------------

%%% Local Variables: 
%%% mode: latex
%%% TeX-master: "../thesis"
%%% End: 
